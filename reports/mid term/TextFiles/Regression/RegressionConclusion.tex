\subsection{Conclusions}

Spiegare MSE

\begin{table}[H]
	\centering
	\begin{tabular}{|| l | r | r | r ||} 
		%\hline
		%\multicolumn{4}{|c|}{Regression metrics} \\
		\hline
		Model & Number of regressors & ResStdErr & $R^2$ \\
		\hline
		Forward stepwise & 13 & 1.124 & 0.9326 \\
		\hline
		Backward stepwise (- 3P MADE) & 12 & 1.124 & 0.9326 \\
		\hline
		Forward stepwise + boost & 13 & 1.124 & 0.9326 \\
		\hline
		Backward stepwise + boost & 12 & 1.124 & 0.9326 \\
		\hline
		Forward stepwise final & 10 & 1.126 & 0.9323 \\
		\hline
		Backward stepwise final & 10 & 1.126 & 0.9323 \\
		\hline
		Ridge regression & 18 & 0.0056 & 0.9998 \\
		\hline
		Lasso regression & 18 & 0.0053 & 0.9998 \\
		\hline
	\end{tabular}
	\caption{RSE and $R^2$ of various models table.}
	\label{table:RegEvalParams}
\end{table}

Based on the model presented in \Tab~\ref{table:ForwardFinalModelSummary}, we can determine that the primary factors that impact a player's scoring ability include the following: minutes played (``MIN''), free throws made (``FTM''), turnovers (``TOV''), assists (``AST''), field goal percentage (``FG\%''), defensive rebounds (``DREB''), three-point percentage (``3P\%''), offensive rebounds (``OREB''), steals (``STL''), and three-point attempts (``3PA'').
Surprisingly, the factor ``GP'' (games played) is not included among these influencing factors.