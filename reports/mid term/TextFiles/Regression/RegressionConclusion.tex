\subsection{Conclusions}

\Tab~\ref{table:RegEvalParams} provides a summary of the results from the application of the four variable selection techniques.
The models exhibit similar performance in terms of MSE and $R^2$. 
However, for our analysis, it is evident that regularization does not contribute significantly to variable selection. 
As a result, we have chosen the model derived from subset selection as our final model.

\begin{table}[h]
	\centering
	\begin{tabular}{|| l | r | r | r ||} 
		\hline
		Model selection algorithm & No. of regressors & MSE & $R^2$(\%) \\
		\hline
		\hline
		Forward stepwise & 10 & 1.1260 & 0.9323 \\
		\hline
		Backward stepwise & 10 & 1.1260 & 0.9323 \\
		\hline
		Ridge regression & 18 & 1.2577 & 0.9379 \\
		\hline
		Lasso regression & 18 & 1.2544 & 0.9380 \\
		\hline
	\end{tabular}
	\caption{Comparison of MSE and $R^2$ values among the previously discussed models.}
	\label{table:RegEvalParams}
\end{table}

Based on the model presented in \Tab~\ref{table:ForwardFinalModelSummary}, we can determine that the primary factors that impact a player's scoring ability include the following: minutes played (``MIN''), free throws made (``FTM''), turnovers (``TOV''), assists (``AST''), field goal percentage (``FG\%''), defensive rebounds (``DREB''), three-point percentage (``3P\%''), offensive rebounds (``OREB''), steals (``STL''), and three-point attempts (``3PA'').
Surprisingly, the factor ``GP'' (games played) is not included among these influencing factors.
