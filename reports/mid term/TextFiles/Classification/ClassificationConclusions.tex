\subsection{Conclusions}

Based on the previous analysis, it was discovered that the variable ``GP'' holds significant importance in explaining the target variable. Interestingly, the variables related to the scored points, such as ``FGM'', ``FTM'' and ``3P MADE'', do not have the high level of significance, as it could be expected. 
However, it is not sufficient by itself to achieve a high level of classification performance.

Indeed, as shown in \Tab~\ref{table:ClasEvalParams}, all the classification methods employed were able to obtain an accuracy around $70\%$. This suggests that there might be other influential factors, not captured by the existing features, that have a substantial impact on determining whether a player will have a lengthy career or not. Thus, it appears that the dataset itself poses a limitation in achieving high performance.

\begin{table}[h]
	\centering
	\begin{tabular}{|| l | r ||} 
		\hline
		Classification technique & MER(\%) \\
		\hline
		\hline
		Logistic regression & 0.3250 \\
		\hline
		Best tree & 0.3000 \\
		\hline
		Bagging & 0.3344 \\
		\hline
		Random forest & 0.2875 \\
		\hline
		Boosting & 0.3031 \\
		\hline
		KNN & 0.3375 \\
		\hline
		SVM & 31.25 \\
		\hline
	\end{tabular}
	\caption{Comparison of MER values among the previously discussed models.}
	\label{table:ClasEvalParams}
\end{table}

