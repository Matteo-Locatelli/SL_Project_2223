The aim of this project is to analyze a dataset collected from NBA Rookies' games. This set contains 1340 samples, each one described by the 21 features listed in table \ref{tab:1}. Each feature, beside \textit{"NAME"}, \textit{"GP"} and \textit{"TARGET\_5YRS"}, is intended to be expressed "per game".
\begin{table}[h]
	\centering
	\small
	\begin{tabular}{||c | c ||}
		\hline
		& Description\\
		\hline
		NAME & Player's name\\ 
		GP & Games Played\\ 
		MIN & Minutes played\\ 
		PTS & Points per Game\\
		FGM & Field Goals Made\\
		FGA & Field Goals Attempts\\
		FG\% & Field Goals Percent\\
		X3P\_MADE & 3 Point Made\\
		X3PA & 3 Point Attempts\\
		X3P\% & 3 Point Percent\\
		FTM & Free Throw Made\\
		FTA & Free Throw Attempts\\
		FT\% & Free Throw Percent\\
		OREB & Offensive Rebounds\\
		DREB & Defensive Rebounds\\
		REB & Rebounds\\
		AST & Assists\\
		STL & Steals\\
		BLK & Blocks\\
		TOV & Turnovers\\
		TARGET\_5YRS & 1 if career duration $\geq 5$, 0 otherwise\\
		\hline
	\end{tabular}
	\caption{Dataset features.}
	\label{tab:1}
\end{table} 

The objective of this analysis is to develop two distinct models that can effectively explain the features \textit{"PTS"} and \textit{"TARGET\_5YRS"}. Due to the nature of these variables, two different approaches are required. For \textit{"PTS"}, a regression method must be utilized due to its continuous nature, whereas for \textit{"TARGET\_5YRS"}, a classification method must be employed given that it is a binary variable.